\begin{resumo}
Este TCC propõe o desenvolvimento de uma ferramenta multiparadigma para operar de forma automatizada no Mercado de Moedas: ferramenta InvestMVC. Cada Paradigma de Programação presente na ferramenta InvestMVC possui seu papel bem definido e suas vantagens de utilização bem esclarecidas. Foram selecionados Métodos Matemáticos para serem implementados na ferramenta através de um Protocolo Experimental bem elaborado. A Qualidade de Software foi uma vertente bem definida no escopo da ferramenta. Testes dinâmicos e estáticos foram estabelecidos. Para que os produtos de trabalho deste TCC fossem elaborados com sucesso, foi definido uma metodologia de pesquisa, com ênfase no objetivo de estudo e nos procedimentos técnicos. Para execução da pesquisa, foi realizada uma adaptação do Scrum, levando em consideração as atividades alocadas no cronograma e o contexto de Engenharia de Software aliado com Mercado de Moedas.

 \vspace{\onelineskip}
    
 \noindent
 \textbf{Palavras-chaves}: Mercado de Moedas, Métodos Matemáticos, Paradigmas de Programação, Qualidade de Software.
\end{resumo}
