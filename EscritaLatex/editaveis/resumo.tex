\begin{resumo}
Este trabalho teve o objetivo desenvolver um software multiparadigma para operar de forma automatizada no Mercado de Moedas: o software InvestMVC. Foi adotada uma metodologia de pesquisa com base nos objetivos e com base nos procedimentos técnicos. Com base nos objetivos foi incorporada uma pesquisa exploratória e descritiva. Em relação aos procedimentos técnicos foi utilizado um estudo de caso, pois apesar deste trabalho ter tido uma abordagem quantitativa dos dados e envolvimento de Métodos Matemáticos para elaboração de estratégias financeiras, não é possível afirmar que os resultados podem ser generalizados para outros contextos, como por exemplo, para outro mercado que não seja o Mercado de Moedas no par de negociação euro-dólar. Para execução da pesquisa, foi realizada uma adaptação do Scrum, levando em consideração as atividades alocadas no cronograma com o desenvolvimento das Histórias de Usuário e suas tarefas. Através do protocolo de estudo de caso, foi definido os passos metodológicos para cumprir os objetivos específicos e responder a questão de pesquisa deste trabalho. Os Métodos Matemáticos de operação do software InvestMVC e a comparação dos resultados monetários com o \textit{expert} MQL foram os resultados mais significativos que auxiliaram a responder a questão de pesquisa. Outros resultados como arquitetura, testes unitários e análise estática de código-fonte, também agregaram bastante valor ao trabalho. Também foi evidenciado atributos do software InvestMVC, como por exemplo, como criar uma conta, fazer login, criar um \textit{expert}, visualizar o \textit{expert} e ativar o mesmo para realizar operações automatizadas. Por fim, foi reportada a conclusão do trabalho.

\vspace{\onelineskip}
    
 \noindent
 \textbf{Palavras-chave}: Mercado de Moedas, Métodos Matemáticos, Paradigmas de Programação, Qualidade de Software.
\end{resumo}
