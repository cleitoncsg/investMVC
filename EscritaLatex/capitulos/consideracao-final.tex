\chapter{Considerações Finais}
Cada Paradigma de Programação presente na ferramenta InvestMVC teve seu papel bem definido através da arquitetura guiada por componentes. Também foi definido que cada componente terão seus testes unitários específicos levando em consideração que cada um é implementado em uma linguagem de programação diferente. O Paradigma Multiagente e Estruturado terão análise estática de código fonte para externalizar a qualidade de código fonte desses paradigmas. A princípio não foi definido análise estática de código para os componentes dos outros paradigmas de programação por causa da ausência de ferramentas que suportem as linguagens de programação desses paradigmas.

Através do Protocolo de Experimentação, os Métodos Matemáticos de Correlação de Pearson, Fibonacci e Mínimos Quadrados foram selecionados para serem implementados na ferramenta InvestMVC. Portanto, espera-se que esses métodos tenham sucesso na ferramenta.

Conforme explanado no capítulo de resultados, a ferramenta InvestMVC encontra-se 25\% finalizada, pois 12 pontos foram concluídos dos 48 pontos definidos no backlog. Qualquer História de Usuário, só é considerada como pronta quando é realizado todos os testes unitários pertinentes. Acredita-se que o percentual de 25\% em breve será aumentado, pois existem várias Histórias de Usuário que estão sendo testadas para serem consideradas finalizadas.
