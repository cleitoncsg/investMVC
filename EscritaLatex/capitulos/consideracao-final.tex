\chapter{Conclusão}
Através do Protocolo de Experimentação, os Métodos Matemáticos de Correlação Linear, Fibonacci e Mínimos Quadrados foram selecionados para serem implementados na ferramenta InvestMVC. 

Cada Paradigma de Programação presente na ferramenta InvestMVC teve seu papel bem definido através da arquitetura guiada por componentes. 

No Componente Estruturado já foram implementados e realizados testes unitários de todos os Métodos Matemáticos (Correlação Linear, Mínimos Quadrados e Fibonacci). Também foi realizada a análise estática de código fonte e a mesma evidencia que os códigos dos Métodos Matemáticos estão excelentes na maioria das métricas. No Componente Funcional foi implementado o método de Correlação Linear e seus respectivos testes unitários. No Componente Multiparadigma, ainda não foi finalizado nenhum agente, mas a estrutura desse Componentes e partes do código do mesmo já estão bem adiantadas. No Componente Lógico, ainda não foi implementado nenhuma linha de código. Mas, a ferramenta de teste unitário PlUnit já foi instalada e testada no ambiente da ferramenta InvestMVC. Por fim, o Componente MQL já foi totalmente implementado e já está em comunicação com a plataforma MetaTrader.

Conforme explanado no capítulo de Resultados, a ferramenta InvestMVC encontra-se 30,61\% finalizada, pois 15 pontos foram concluídos dos 49 pontos definidos no backlog. Qualquer História de Usuário, só é considerada como pronta quando é realizado todos os testes unitários pertinentes. Acredita-se que o percentual de 30,61\% em breve será aumentado, pois existem várias Histórias de Usuário que estão sendo testadas para serem consideradas finalizadas.

A mistura harmônica de Mercado de Moedas, Métodos Matemáticos, Paradigmas de Programação e Qualidade de Software está gerando a grande ferramenta InvestMVC. Espera-se no TCC 2, uma evolução bastante significativa da  ferramenta para que este trabalho não se torne apenas um trabalho de consulta na biblioteca da UnB, mas sim um trabalho que será utilizado pela sociedade.
