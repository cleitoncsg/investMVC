\chapter{Conclusão}
\label{sec:conclusao}

Este trabalho teve o objetivo de desenvolver o software InvestMVC que, por sua vez, é um software multiparadigma que realiza operações financeiras (compra ou venda) no Mercado de Moedas de forma automatizada. Sendo assim, é possível que um usuário que não conheça o Mercado de Moedas, opere através do software e ganhe dinheiro de acordo com a(s) estratégia(s) selecionada(s). Para que o objetivo geral fosse alcançado, foram definidos os objetivos específicos (produtos de trabalho). O primeiro e último objetivo específico (selecionar Métodos Matemáticos e comparar resultados financeiros) tiveram maior peso neste trabalho, pois através dos mesmos foi possível definir passos metodológicos para obter os resultados monetários dos Métodos Matemáticos e responder a questão de pesquisa deste trabalho.

A metodologia foi definida com base nos objetivos e com base nos procedimentos técnicos. Com base nos objetivos foi incorporada uma pesquisa exploratória e descritiva. Em relação aos procedimentos técnicos, foi utilizado um Estudo de Caso, pois apesar deste trabalho ter tido uma abordagem quantitativa dos dados e envolvimento de Métodos Matemáticos para elaboração de estratégias financeiras, não é possível afirmar que os resultados podem ser generalizados para outros contextos, como por exemplo, para outro mercado como a bolsa de valores. O rigor metodológico para o Estudo de Caso foi estabelecido através do planejamento do Estudo de Caso, projeto, coleta dos dados, análise dos dados, validade do plano e limitações de estudo. Outros atributos atrelados a metodologia também foram definidos como atividades da pesquisa, execução da pesquisa e cronograma. A execução da pesquisa não ocorreu conforme o que foi planejado no cronograma e nem todas as atividades da pesquisa foram agregadas ao trabalho. Ajustes foram necessários no decorrer do processo de desenvolvimento do trabalho para que os objetivos específicos fossem alcançados.

Cada Paradigma de Programação presente no software InvestMVC teve seu papel bem definido na da arquitetura guiada por componentes. No componente Estruturado foram implementados e realizados testes unitários de todos os Métodos Matemáticos (Correlação Linear, Mínimos Quadrados e Fibonacci). No componente Funcional foi implementado e testado os mesmos Métodos Matemáticos do componente Estruturado. Desta forma, o paradigma Funcional e Estruturado trabalharam juntos nos mesmos cálculos matemáticos e geram um mecanismo de tolerância a falhas. No componente Lógico, foi implementada e testada a inserção dos dados das operações financeiras (tipo de ordem, tipo de método usado, resultado da operação financeira). No componente Multiagente, foram implementados e testados os diversos agentes com suas responsabilidades específicas. Além disso, foi elaborada a comunicação desse paradigma com os demais paradigmas do software InvestMVC. Por exemplo, o componente Multiagente recebe a resposta dos cálculos numéricos dos componentes Estruturado e Funcional e verificam se é possível emitir uma ordem de compra ou venda para o componente MQL. Por fim, o Componente MQL foi implementado para realizar as operações de compra ou venda de acordo com a ordem do componente Multiagente, porém não foi possível implementar testes unitários para o mesmo, pois a linguagem MQL não possui nenhum mecanismo que permita realizar testes unitários.

Foi necessário selecionar Métodos Matemáticos para serem implementados no software InvestMVC. Então, foi proposto o objetivo específico de “selecionar Métodos Matemáticos para serem implementados no software InvestMVC”. Através da metodologia de pesquisa, foram definidos passos para se alcançar esse objetivo específico. Depois de realizar simulações dos métodos, foi selecionado o método de Correlação de Pearson, Fibonacci e Mínimos Quadrados para serem implementados no software InvestMVC.

Para desenvolver o software InvestMVC, seria necessário analisar as arquiteturas existentes dos paradigmas de programação e estudar como os paradigmas poderiam se comunicar de forma harmônica. Além disso, era necessário explorar o que cada paradigma poderia oferecer de melhor. Então foi elaborado o objetivo específico “caracterizar as estruturas e componentes do software InvestMVC”. A arquitetura desenvolvida neste trabalho foi guiada por componentes e cada componente possui um paradigma de programação com suas responsabilidades específicas.

Para obter a qualidade do software InvestMVC, seria viável propor testes unitários e realização de análise estática de código-fonte. Então, foi proposto os objetivos específicos “realizar análise estática de código-fonte do paradigma multiagente a partir de métricas previamente definidas” e “apurar a cobertura de código por meio de ferramentas que implementam testes unitários nos Paradigmas Estruturado (linguagem C), Multiagentes (linguagem Java), Lógico (linguagem Prolog) e Funcional (linguagem Haskell)”. Foram realizados testes unitários de todos os componentes do software InvestMVC, exceto o componente MQL, pois a linguagem MQL não possui nenhuma ferramenta para desenvolvimento de testes unitários. A cobertura de código-fonte foi de 100\% no componente Estruturado e acima de 80\% no componente Multiagente, contudo não foi possível obter a cobertura dos componentes Funcional e Lógico, apesar de terem sido desenvolvidos os testes unitários. Em relação a análise estática de código-fonte, foi obtido a qualidade de código-fonte do componente Multiagente e a qualidade ficou em um nível aceitável. A maioria das métricas deram resultado excelente.

Foi mostrado neste trabalho, como pode ser utilizado o software InvestMVC, desde a criação de um usuário, \textit{login}, criação de um \textit{expert}, ativação do mesmo e visualização dos resultados. Também foi mostrado como um usuário que não tenha conhecimento sobre o Mercado de Moedas, pode ser auxiliado através da opção de “Suporte”. 

Para responder a questão de pesquisa deste trabalho, seria necessário colocar o software InvestMVC e o \textit{expert} MQL para operar no Mercado de Moedas e obter os resultados monetários. Então, foi desenvolvido o objetivo específico “comparar resultados monetários do software InvestMVC e do \textit{expert} implementado em linguagem MQL”. Foram selecionados os métodos de operação, tempo de operação, tipo de gráfico e demais atributos para os produtos de software operarem conforme estabelecido na metodologia de pesquisa. Os resultados do software InvestMVC e do \textit{expert} MQL foram semelhantes. Os pontos de entrada e saída no mercado não ficaram iguais, porém próximos. Com isso, os valores monetários também ficaram próximos. O software InvestMVC obteve pouco mais de 5\% de lucro em relação ao \textit{expert} MQL.

Sugere-se como trabalhos futuros, analisar os tempos de resposta do \textit{expert} implementado em linguagem MQL e do software InvestMVC. Sugere-se também, analisar os pontos de entrada e saída das operações de compra e venda. Por fim, em relação a qualidade do produto, sugere-se realizar outros tipos de teste como:  teste funcional e teste de integração entre os componentes.

Neste trabalho foi proposta a seguinte questão de pequisa: “é possível que o software InvestMVC substitua os \textit{experts} tradicionais do Mercado de Moedas?”. Diante dos resultados deste deste trabalho, oriundos dos objetivos específicos, há fortes indícios que sim, o software InvestMVC pode ser uma alternativa aos \textit{experts} tradicionais utilizados para operarem no Mercado de Moedas.

%A expectativa dos autores é que o software InvestMVC possa ser usado por pessoas que tenham conhecimento do Mercado de Moedas ou até mesmo que não entendam muito sobre este mercado. Sendo assim, espera-se que este trabalho agregue valor a sociedade.

A expectativa dos autores é que o software InvestMVC possa ser usado por pessoas que tenham conhecimento do Mercado de Moedas ou até mesmo que não entendam muito sobre este mercado. Sendo assim, espera-se que este trabalho incentive pessoas a buscarem conhecimento sobre este mercado e as aplicações da Engenharia de Software neste contexto.