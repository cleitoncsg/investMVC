\chapter{Glossário}

\textbf{Cotação} - valor monetário do par de moedas em um determinado período de tempo.

\textbf{Corretagem} - Cobrança em valor monetário por alguma operação de compra ou venda.

\textbf{Corretora} - Broker que intermedia a compra ou venda no Mercado de Moedas

\textbf{CSV} - Extensão que permite manipular arquivos em planilhas eletrônicas.

\textbf{Expert} - Software que realiza operações de compra ou venda no Mercado de Moedas.

\textbf{FGA} - Faculdade do Gama

\textbf{Framework} - Captura a funcionalidade comum a várias aplicações

\textbf{GPL3} - Permite que os programas sejam distribuídos e reaproveitados

\textbf{LGPL3} -  Permite que os programas sejam distribuídos e reaproveitados com outras licenças que não sejam GPL ou LGPL.

\textbf{Middleware} - Software que atua como um mediador, entre dois programas existentes e independentes.

\textbf{MQL4} - Linguagem de programação em paradigma estruturado que permite implementar experts.

\textbf{MQL5 }- Linguagem de programação em paradigma orientado a objetos que permite implementar experts.

\textbf{Stop loss} - Define a quantidade de pontos que o investidor deseje perder.

\textbf{Take profit} - Define a quantidade de pontos que o investidor deseje ganhar.

\textbf{Tendência} - Revela a direção em que o mercado está indo.

\textbf{USD} - Dólares americanos

\textbf{UnB} - Universidade de Brasília
