\chapter{Estado Atual}


\begin{table}[htp]
\caption{Atividades e objetivos metodologia de pesquisa}
\begin{center}
    \begin{tabular}{ | p{5cm} | p{4cm} | p{4cm}|}
    \hline
    \textbf{Histórias de Usuário} & \textbf{Pontuação} \\ \hline

US1 - Agente Correlação Linear & 3\\ \hline
US2 - Agente Fibonacci & 3 \\ \hline
US3 - Agente Mínimos Quadrados & 3\\ \hline
US4 -  Agente Tendência & 2 \\ \hline
US5 - Agente Gestor/Consultor & 2\\ \hline
US6 - Criar conta de usuário & 2\\ \hline
US7 - Acompanhar retorno financeiro & 5\\ \hline
US8 - Criar Experts & 2\\ \hline
US9 - Editar Experts & 2\\ \hline
US10 - Excluir Experts & 2\\ \hline
US11 - Ativar Expert & 1\\ \hline
US12 - Desativar Expert & 2 \\ \hline
US13 - Método Correlação Linear em linguagem C & 2\\ \hline
US14 - Método Fibonacci em linguagem C & 2\\ \hline
US15 - Método Mínimos Quadrados em linguagem C & 2\\ \hline
US16 - Método Correlação Linear em linguagem Haskell & 2\\ \hline
US17 - Método Fibonacci em linguagem Haskell & 2\\ \hline
US18 - Método Mínimos Quadrados em linguagem Haskell & 2\\ \hline
US19 - Inserir na Base de Conhecimento & 2\\ \hline
US20 - Retirar na Base de Conhecimento & 2\\ \hline
US21 - Calcular Critério de Entrada & 3\\ \hline
\textbf{Total de pontos} & 48\\ \hline
\end{tabular}
\end{center}
\label{estadoAtual}
\end{table}
