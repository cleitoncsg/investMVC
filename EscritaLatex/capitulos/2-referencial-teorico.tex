\chapter{Referencial teórico}

O referencial teórico revela o momento de levantar o embasamento teórico sobre o tema de pesquisa. No contexto desse TCC, faz-se necessário, dentros outros aspectos pesquisar sobre os entendimentos existentes do problema de pesquisa e analisar quais mecanismos devem ser adotados para se propor uma solução \cite{belchior2012}.

O referencial teórico deste TCC irá levantar o embasamento teórico sobre Contexto Financeiro, Métodos Matemáticos, Paradigmas de Programação e Testes estáticos/dinâmicos para que seja possível propor uma solução para o problema de pesquisa.

\section{Contexto Financeiro}
Esta seção irá tratar os atributos aliados ao contexto financeiro como Alavancagem, Suporte e Resistência. Esses atributos são insumos para que se possa compreender melhor a dinâmica das estratégias para negociação no Mercado de Moedas.

\subsection{Mercado de Moedas}
Mercado de Moedas ou FOREX (abreviatura de Foreign Exchange) é um mercado interbancário onde as várias moedas do mundo são negociadas. O FOREX foi criado em 1971, quando a negociação internacional transitou de taxas de câmbio fixas para flutuantes. Com o resultado do seu alto volume de negociações, o Mercado de Moedas tornou-se o principal mercado financeiro do mundo \cite{market2011}.

A operação no Mercado de Moedas envolve a compra de uma moeda e a simultânea venda de outra. As moedas são negociadas em pares, por exemplo: euro e dólar (EUR-USD). O investidor não compra ou vende euro e dólares fisicamente, mas existe uma relação monetária de troca entre eles. O FOREX é um mercado em que são negociados, portanto, derivativos de moedas. O investidor é remunerado pelas diferenças entre a valorização (se tiver comprado) ou desvalorização (se tiver vendido) destas moedas \cite{cvm2009}.

O Mercado de Moedas é descentralizado, pois as operações são realizadas por vários participantes do mercado em vários locais. É raro uma moeda manter uma cotação constante em relação a outra moeda. O câmbio entre duas moedas muda constantemente \cite{fxcm2011}.

O Mercado de Moedas é constituído por transações entre as corretoras que operam no mesmo e são negociados, diariamente, contratos representando volume total entre 1 e 3 trilhões de dólares. As transações são realizadas diretamente entre as partes (investidor e corretora) por telefone e sistemas eletrônicos, desde que tenham conexão à internet. As operações ocorrem 24 horas por dia, durante 5 dias da semana (abrindo às 18h no domingo e fechando às 18h na sexta; horário de Brasília), negociando os principais pares de moedas, ao redor do mundo \cite{cvm2009}.

\subsection{Alavancagem}
