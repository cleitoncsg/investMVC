\chapter{Introdução}
Este capítulo aborda o contexto, o problema de pesquisa, a justificativa, os objetivos e como o trabalho está organizado.

\section{Contextualização}
O Mercado de Moedas é constituído por transações entre as corretoras que operam no mesmo e são negociados, diariamente, contratos representando volume total entre 1 e 3 trilhões de dólares. É possível operar nesse mercado de forma manual ou automatizada \cite{cvm2009}.

No contexto de operações automatizadas, no Mercado de Moedas estão inseridos os produtos de software conhecidos como Experts, que processam métodos numéricos para gerar critérios de entrada e saída para comprar ou vender nesse mercado.

\section{Problema de Pesquisa}
Este TCC procurará responder a seguinte questão: é possível desenvolver uma ferramenta multiparadigma que substitua os Experts convencionais do Mercado de Moedas? 

\section{Justificativa}

Operar no Mercado de Moedas de forma manual é muito arriscado e, portanto, não recomendado, uma vez que esse mercado é não previsível, o que pode provocar a perda do capital de um investidor em apenas alguns minutos. Em diversas situações, o mercado varia as cotações em apenas um minuto, sendo que a mesma variação pode ser feita durante horas. Para contornar esse problema, a plataforma MetaTrader\footnote{\url{http://www.metatrader4.com/}} oferece as linguagens MQL4 (Paradigma Estruturado) e MQL5 (Paradigma Orientado a Objetos) para construir Experts que operem de forma automatizada. 

A plataforma MetaTrader não oferece suporte de ferramentas de teste unitário para as linguagens MQL4 e MQL5. Após implementar um Expert, não é possível implementar testes de unidade para verificar se as instruções programadas estão de acordo com o esperado. A única forma de verificar se o Expert está seguindo as estratégias programadas é usar uma conta real ou demo na plataforma e operar durante um período específico de tempo.

Não foi possível encontrar na literatura investigada até o momento ferramentas que realizem a análise estática de código fonte em MQL4 e MQL5. Portanto, torna-se difícil obter uma análise de critérios de aceitabilidade (ou orientada a métricas) no nível de código fonte dos Experts programados nessas linguagens.

Adicionalmente, o código da plataforma MetaTrader é fechado. Dessa forma, não é possível a colaboração da comunidade de desenvolvedores no que tange a evolução das funcionalidades da ferramenta anteriormente.

Diante das preocupações acordadas até o momento, acredita-se que o desenvolvimento de uma ferramenta de código aberto para investimento no Mercado de Moedas, orientada a modelos conceituais de diferentes Paradigmas de Programação bem como às boas práticas da Engenharia de Software como um todo, irá conferir ao investidor maior segurança e conforto em suas operações. Portanto, a ferramenta proposta será implementada em diferentes linguagens de programação, padrões de projeto adequados, testes unitários orientados à uma abordagem multiparadigmas e análise qualitativa de código fonte. Um Expert (implementado em linguagem MQL4 ou MQL5) ou um conjunto de Experts serão substituídos pela ferramenta. Nesse último caso, a ferramenta terá a propriedade de controlar e/ou monitorar um ou mais Experts.

\section{Objetivos}

\subsection{Objetivo Geral}

Desenvolver uma ferramenta multiparadigma  para operar de forma semi-automatizada no Mercado de Moedas.


\subsection{Objetivos Específicos}

Considerando o Mercado de Moedas e os Paradigmas de Programação Estruturado,  Orientado a Objetos,  Funcional, Lógico  e  Multiagentes, são objetivos específicos desse TCC:

\begin{enumerate}
\item Selecionar Métodos Matemáticos para serem implementados na ferramenta MVC através de um protocolo de experimentação;

\item Caracterizar as implementações dos códigos da ferramenta MVC através dos Paradigmas  de  Programação;

\item Comparar  resultados financeiros obtidos pelo código no Paradigma Estruturado com códigos produzidos nos demais Paradigmas;

\item Realizar análise estática do código fonte dos produtos de software a partir de métricas de qualidade previamente definidas;

\item Apurar a cobertura de código por meio de ferramentas que implementam testes unitários nos Paradigmas Estruturado (linguagem C), Multiagentes (linguagem Java), Lógico (linguagem Prolog) e Funcional (linguagem Haskell);

\item Desenvolver testes de integrações e funcionais para a ferramenta MVC.
\end{enumerate}

\section{Organização do Trabalho}

No capítulo 2, é apresentado o Referencial Teórico quanto ao Contexto Financeiro, Métodos Numéricos e aos Paradigmas de Programação. No Contexto Financeiro são tratados os atributos atrelados ao Mercado de Moedas como alavancagem, suporte e resistência. Em Métodos Numéricos, é realizada uma descrição dos métodos de Fibonacci, Correlação de Pearson e Mínimos Quadrados. Em Paradigmas de Programação são descritos os paradigmas: Estruturado, Orientado a Objetos, Lógico, Funcional e Multiagentes. Em Testes de Software, são evidenciados quais os tipos de testes serão utilizados neste TCC e também quais linguagens terão testes. Por fim, em Qualidade de Software são externalizadas as métricas de qualidade de  código fonte, critérios para interpretação das métricas e ferramentas de análise estática.
