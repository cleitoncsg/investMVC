\chapter{Introdução}

Operar no mercado de moedas de forma manual é muito arriscado e, portanto, não recomendado, uma vez que esse mercado é não previsível, o que pode provocar a perda do capital de um investidor em apenas alguns minutos. Em diversas situações, o mercado varia as cotações em apenas um minuto, sendo que a mesma variação pode ser feita durante horas. Para contornar esse problema, a plataforma MetaTrader\footnote{\url{http://www.metatrader4.com/}} oferece as linguagens MQL4 (Paradigma Estruturado) e MQL5 (Paradigma Orientado a Objetos) para construir \textit{Experts} que operem de forma automatizada. 

A plataforma MetaTrader não oferece suporte de ferramentas de teste unitário para as linguagens MQL4 e MQL5. Após implementar um \textit{Expert}, não é possível implementar testes unitários para verificar se as instruções programadas estão de acordo com o esperado. A única forma de verificar se o \textit{Expert} está seguindo as estratégias programadas é usar uma conta real ou demo na plataforma e operar durante um período específico de tempo.

Não foi possível encontrar na literatura investigada até o momento ferramentas que realizem a análise estática de código fonte em MQL4 e MQL5. Portanto, torna-se difícil obter uma análise de critérios de aceitabilidade (ou orientada a métricas) no nível de código fonte dos \textit{Experts} programados nessas linguagens.

Adicionalmente, o código da plataforma MetaTrader é fechado. Dessa forma, não é possível a colaboração da comunidade de desenvolvedores no que tange a evolução das funcionalidades do MetaTrader.

Diante das preocupações acordadas até o momento, acredita-se que o desenvolvimento de um software de código aberto para investimento no Mercado de Moedas, orientado a modelos conceituais de diferentes Paradigmas de Programação bem como às boas práticas da Engenharia de Software como um todo, irá conferir ao investidor maior segurança e conforto em suas operações. Portanto, o software proposto será implementado em diferentes linguagens de programação, padrões de projeto adequados, testes unitários orientados à uma abordagem multiparadigmas e análise qualitativa de código fonte. Um Expert (implementado em linguagem MQL4 ou MQL5) ou um conjunto de Experts serão substituídos pelo software. Nesse último caso, o software terá a propriedade de controlar e/ou monitorar um ou mais \textit{Experts}.

Por fim, este trabalho procurará responder a seguinte questão de pesquisa: é possível desenvolver um software multiparadigma que substitua os Experts convencionais do Mercado de Moedas?

\section{Objetivos}
Este trabalho tem como objetivo geral desenvolver o software multiparadigma InvestMVC que utiliza métodos matemáticos para automação de estratégias financeiras no mercado FOREX.

Considerando o Mercado de Moedas e os Paradigmas de Programação Estruturado, Orientado a Objetos, Funcional, Lógico e Multiagentes, são objetivos específicos deste trabalho:

\begin{enumerate}
\item  Selecionar métodos matemáticos a serem implementados no software InvestMVC com utilização de um protocolo de experimentação;

\item Caracterizar as estruturas e componentes do software InvestMVC;

\item Realizar análise estática do código fonte dos produtos de software a partir de métricas de qualidade previamente definidas;

\item  Apurar a cobertura de código por meio de ferramentas que implementam testes unitários nos Paradigmas Estruturado (linguagem C), Multiagentes (linguagem Java),Lógico (linguagem Prolog) e Funcional (linguagem Haskell);

\item Desenvolver testes de integração e funcionais para a software InvestMVC;

\item Comparar resultados financeiros obtidos pelo software InvestMVC com os \textit{Experts} tradicionais implementados em linguagem MQL;
\end{enumerate}

\section{Organização do Trabalho}
No capítulo 2, é apresentado o Referencial Teórico quanto ao Contexto Financeiro, métodos matemáticos, Paradigmas de Programação, Teste e Qualidade de Software. No Contexto Financeiro, são tratados os atributos atrelados ao Mercado de Moedas como Alavancagem, Suporte e Resistência. Em métodos matemáticos, é realizada uma descrição dos métodos de Fibonacci, Correlação de Pearson, Mínimos Quadrados, Estocástico e Média Móvel. Em Paradigmas de Programação, são descritos os paradigmas: Estruturado, Orientado a Objetos, Lógico, Funcional e Multiagentes. Em Testes de Software, são evidenciados quais os tipos de testes serão utilizados neste trabalho e também quais linguagens terão testes. Por fim, em Qualidade de Software são externalizadas as métricas de qualidade de código fonte, critérios para interpretação das métricas e ferramentas de análise estática. No Capítulo 3, é apresentada a Metodologia de Pesquisa e seus atributos como Classificação da Pesquisa, Atividades da Pesquisa, Execução da Pesquisa, Planejamento do Protocolo de Experimentação e o Cronograma. No Capítulo 4, são apresentados os Resultados com base nos objetivos específicos do trabalho. Por fim, no Capítulo 5 são apresentadas as Conclusões.
