\chapter{Trabalhos Futuros}

Foram realizados testes unitários dos paradigmas Funcional, Lógico, Estruturado e Multiagentes. O componente Multiagente teve mais de 80\% de cobertura de código. Entretanto, os métodos \textit{actions} foram ignorados na obteção da cobertura de teste, pois os agentes de software necessitam de um tempo de resposta para outro agente se comunicar. Apesar terem sido colocados \textit{delays}  nos testes unitários utilizando o EasyMock e o Junit, não foi possível contornar esse problema. Também não foi possível utilizar a ferramenta do próprio \textit{framework} Jade para teste. Sugere-se um estudo para obtenção de testes unitários aplicáveis utilizando o paradigma Multiagente (linguagem Java).

Foi realizada a análise estática de código-fonte do paradigma Multiagente, pois métricas OO são métricas mais bem consolidadas na literatura. Recomenda-se investigar se a análise estática no paradigma Multiagente utilizando a linguagem Java, realmente é próxima de uma análise estática de um código no paradigma OO, também utilizando a linguagem Java. Recomenda-se também, um estudo para sugerir possíveis melhorias de refatoração no código depois da obtenção de uma métrica não aceitável (ruim ou péssima, por exemplo).

Para selecionar os Métodos Matemáticos que foram implementados no software InvestMVC, foram escolhidos de forma empírica 5 métodos: Média Móvel, Estocástico, Mínimos Quadrados, Correlação Linear e Fibonacci. Depois de percorrer critérios de seleção, os métodos de Média Móvel e Estocásticos foram descartados. Recomenda-se estudar mais Métodos Matemáticos e realizar simulações dos mesmos para possivelmente aumentar o índice de acerto nas operações do software InvestMVC.

Na comparação de valores monetários do software InvestMVC e do \textit{expert} implementado em linguagem MQL, observou-se que o software InvestMVC realiza algumas operações no mercado em um horário diferente do expert MQL, apesar de ambos produtos de software terem as mesmas configurações conforme estabelecido na metodologia de pesquisa. Os pontos de entrada também foram diferentes. O desvio padrão dos horários de entrada foi de 4.23 segundos e a Correlação de Pearson dos horários de entrada foi de 0.83 (correlação moderada). Recomenda-se investigar o motivo de um desvio tão alto e de uma correlação moderada. Recomenda-se também investigar o motivo dos produtos de software entrarem em pontos diferentes. 

Por fim, recomenda-se aos interessados executar o software InvestMVC durante um período de tempo extenso para baixar o histórico de operações e analisar os resultados da operações.
